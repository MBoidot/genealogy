\documentclass[12pt,a4paper]{article}

\usepackage[T1]{fontenc}
\usepackage[utf8]{inputenc}
\usepackage{geometry}
\geometry{margin=2.5cm}

% Police élégante, très lisible (style garalde / médiévale légère)
\usepackage{mathpazo} % Palatino, très lisible et historique

% Lettrines décorées
\usepackage{lettrine}

% Cadres ornementaux sobres
\usepackage{tcolorbox}
\tcbset{
  sharp corners,
  colback=white,
  colframe=black,
  boxrule=0.8pt,
  left=5mm,
  right=5mm,
  top=3mm,
  bottom=3mm,
}

% TikZ pour décorations de page
\usepackage{tikz}
\usepackage{eso-pic}
\usetikzlibrary{calc} % nécessaire pour les coordonnées calculées
\usepackage{everypage}

% Petits symboles sûrs
\usepackage{amssymb}  % fournit \lozenge si besoin
\usepackage{ragged2e} % pour \justifying

% Hook de décor de page — propre et sans constructions fragiles
\AddToShipoutPictureBG*{%
  \begin{tikzpicture}[remember picture,overlay]
    % Cadre principal
    \draw[line width=0.8pt]
      ($(current page.north west)+(1cm,-1cm)$)
        rectangle
      ($(current page.south east)+(-1cm,1cm)$);

    % Ornements simples
    \node at ($(current page.north west)+(1.2cm,-0.8cm)$) {\small$\blacklozenge$};
    \node at ($(current page.north east)+(-1.2cm,-0.8cm)$) {\small$\blacklozenge$};
    \node at ($(current page.south west)+(1.2cm,0.8cm)$) {\small$\blacklozenge$};
    \node at ($(current page.south east)+(-1.2cm,0.8cm)$) {\small$\blacklozenge$};
  \end{tikzpicture}
}

% Lignes resserrées
\newcommand{\ligne}{\vspace{0.35cm}\hrule}

% Titres élégants
\newcommand{\SectionTitle}[1]{%
  {\centering\Large\bfseries #1\par}
}

\begin{document}

% ============================================================
% INTRODUCTION
% ============================================================

\SectionTitle{Introduction}

\begin{lettrine}[lines=3,lhang=0.3]{D}{ans} cette démarche familiale, nous cherchons à rassembler les récits, souvenirs et témoignages de chaque membre afin de reconstruire une mémoire commune et vivante. 
Ces pages ont pour vocation de révéler les histoires personnelles, les moments marquants, les traditions, les liens tissés et les expériences qui font l’identité de notre famille. 
Chaque contribution, qu’elle soit longue ou brève, structurée ou spontanée, constitue une pièce essentielle de ce patrimoine intime.
Votre participation permettra de transmettre à la génération suivante non seulement des faits, mais aussi des émotions, des valeurs et une vision humaine de notre histoire.
Merci de contribuer à cette œuvre collective en répondant aux thèmes proposés ou en partageant librement ce qui vous semble important.
\end{lettrine}

\vspace{1cm}

% Contacts
\begin{tcolorbox}
\centering
\textbf{Pour transmettre vos réponses :}\par
mathieu.boidot@gmail.com\par
francois.boidot@nordnet.fr
\end{tcolorbox}

\vspace{1cm}

% Cadre méta-civiles
\begin{tcolorbox}[title={Informations méta-civiles}]
\noindent\textbf{Nom :} \ligne\par
\noindent\textbf{Prénom :} \ligne\par
\noindent\textbf{Date de naissance :} \ligne\par
\noindent\textbf{Lieu de naissance :} \ligne\par
\noindent\textbf{Téléphone :} \ligne\par
\noindent\textbf{Mail :} \ligne\par
\end{tcolorbox}

\newpage

% ============================================================
% THEMES
% ============================================================

\SectionTitle{Thèmes à explorer}

% Template d'un thème
\newcommand{\Theme}[2]{%
\vspace{0.8cm}
{\large\bfseries #1}\par\vspace{2mm}
\begin{tcolorbox}
\justifying
#2
\end{tcolorbox}
% 10 lignes resserrées
\noindent\ligne\par\noindent\ligne\par\noindent\ligne\par\noindent\ligne\par\noindent\ligne\par
\noindent\ligne\par\noindent\ligne\par\noindent\ligne\par\noindent\ligne\par\noindent\ligne\par
\vspace{0.8cm}
}

% Thèmes
\Theme{1. Origines et enfance}{
\textit{Évoquez vos premières années, le contexte familial, vos souvenirs les plus anciens, l'ambiance de la maison, les moments marquants de votre enfance.}
}

\Theme{2. Vie familiale}{
\textit{Décrivez votre relation avec vos parents, frères, sœurs, ou proches. Parlez des traditions familiales, des habitudes, ou de ce qui vous a marqué dans la vie de famille.}
}

\Theme{3. Parcours personnel et professionnel}{
\textit{Racontez votre scolarité, votre orientation, vos passions, vos métiers, les choix importants qui ont façonné votre vie.}
}

\Theme{4. Moments marquants}{
\textit{Qu'il s'agisse de réussites, de difficultés, de rencontres décisives ou de changements de vie : décrivez les événements qui vous ont profondément marqué.}
}

\Theme{5. Expression libre}{
\textit{Toute réflexion, souvenir, sentiment ou anecdote que vous souhaitez transmettre, sans contrainte ni orientation particulière.}
}

\end{document}
